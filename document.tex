\documentclass{article}
\usepackage[indentfirst, chemistry, maths, flux]{taiga}
\usepackage[utf8]{inputenc}

\title{Taiga}
\author{Toby Lam}
\date{February 2021}

\begin{document}

\maketitle

\section{Introduction}

A personal custom LaTeX Package that contains various other packages for more convenient use

\section{Packages required}

\subsection{ifthen}

So that I could use some better if then code for package options implementation. May be reductant

\subsection{[T1] fontenc, [utf8] inputenc}

Just standard font encoding stuff idk

\subsection{Graphicx}

Put Images, i.e.

\begin{verbatim}
	\includegraphics[width=\textwidth]{universe}
\end{verbatim} 

\subsection{[hyphens] url}

So that hyphens could wrap around urls properly. Must be put before hyperref

\subsection{url}

Produce hypertext links in the document, i.e.

\begin{verbatim}
	\href{google.com}{Google}
\end{verbatim} 

\subsection{multicol}

Multiple columns, i.e.

\begin{verbatim}
	\begin{multicols}{2}[ %Text with 1 column]	
		%Text to be seperated automatically by multicol 
	\end{multicols}
\end{verbatim}

\subsection{lipsum}

Sample text, i.e.

\begin{verbatim}
	\lipsum[2-4]
\end{verbatim}

\subsection{todonotes}

\begin{verbatim}
	\todo{Add details} 
\end{verbatim}

\subsection{textcomp}

Provides extra symbols, i.e.

\begin{verbatim}
	\textrightarrow, \textcelsius
\end{verbatim}

\subsection{caption}

Captions in tables

\subsection{gensymb}

Provides generic commands which work on both text / math mode, i.e.

\begin{verbatim}
	\degree, \celsius, \perthousand, \micro and \ohm
\end{verbatim}

\subsection{booktabs}

\begin{verbatim}
	\begin{tabular}{@{}llr@{}}
	\toprule
	\addlinespace[0.1em]
	\middlerule
	\bottomrule
	\end{tabular}
\end{verbatim}

\subsection{float}

For tables so that the location is precise
	
\begin{verbatim}
	\begin{table}[H]
\end{verbatim}


\subsection{microtype}

Make stuff look better?

\subsection{siunitx}

\begin{verbatim}
	\si{kg.m.s^{-1}} or \si{\kilogram\metre\per\second}
	\num{.3e45} 
\end{verbatim}

\subsection{times}

Times font. I just really like it.

\section{Options}

\subsection{maths}

Uses tikz, pgf and pgfplots to plot graphs. Below is required
\begin{verbatim}
	pgfplotsset{compat=1.15} 
\end{verbatim}

Uses all of ams packages plus mathtools which seems to makes things nicer

The below means that the Y counter restarts every X.

\begin{verbatim}
	newtheorem{Y}{Y}[X]
\end{verbatim}

Here's how to use the environments for $Y \in \{ \text{theorem, lemma, definition} \}$

\begin{verbatim}
	\begin{Y}[Name of the thing]
	\end{Y}
\end{verbatim}

There are also some nice shortcuts 

\begin{verbatim}
	\N: \mathbb{N}
	\Z: \mathbb{Z}
	\Q: \mathbb{Q}
	\R: \mathbb{R}
\end{verbatim}

\subsection{chemistry}

mcchem for displaying chemical formulae

chemfig and tikz for drawing molecular structures

\subsection{indentfirst}

So that every paragraph is indented

\subsection{flux}

Puts a yellow tint over the entire document

\subsection{bibliography}

To use, create a bibliography.bib file and edit it using JabRef
Site stuff by using 
\begin{verbatim}
	\cite{}
\end{verbatim}
At the end of the document, add 
\begin{verbatim}
	\bibliography{bibliography}
\end{verbatim}

\subsection{Legacy}

Many of my old documents use 

\begin{verbatim}
	\ce{->}
\end{verbatim}

So I included mcchem

\subsection{a4paper}

Uses geometry to make the margins more suited for a4

\subsection{noindent}

No indent for all paragraphs

\subsection{tocnosections}

Only parts and chapters would be displayed in table of content

\end{document}
